\documentclass[a4paper,15pt]{article}
\usepackage{hyperref}
\usepackage{amsmath}
\usepackage{geometry}
\usepackage{fancyhdr}
\usepackage{enumitem}
\usepackage{tocbibind} % To include "Table of Contents" in the index
\geometry{margin=1in}
\pagestyle{fancy}
\fancyhf{}
\fancyhead[L]{Lab Notebook - Team 34}
\fancyhead[R]{\thepage}

\title{\textbf{LAB NOTEBOOK}}
\author{\textbf{Team 34}}
\date{}

\begin{document}

\maketitle


\begin{center}
    \Large\textbf{Maulana Abul Kalam Azad University of Technology}\\
    \vspace{0.2cm}
    \large Software Tools and Technology - Lab Notebook
\end{center}

\vspace{1cm}

\section*{Assignment Details}
\begin{itemize}[leftmargin=1.5cm]
    \item \textbf{Assignment:} Create a Git Repository Containing a Lab Notebook in LaTeX Format
    \item \textbf{Subject:} Software Tools and Technology
    \item \textbf{Team No.:} 34
    \item \textbf{GitHub Repo Link:} \url{https://github.com/Srinjay17/Group-34.git}
\end{itemize}

\vspace{1cm}

\section*{Team Members}
\begin{itemize}[leftmargin=1.5cm]
    \item \textbf{Member 1 (Lead):} 
    \begin{itemize}[leftmargin=1.5cm]
        \item \textbf{Name: Srinjay Nag} 
        \item \textbf{Roll No: 30059223037}
        \item \textbf{Department: Forensic Science and Technology (BSc Forensic Science)} 
        \item \textbf{GitHub Link:} \url{https://github.com/Srinjay17}
\end{itemize}

\item \textbf{Member 2:} 
    \begin{itemize}[leftmargin=1.5cm]
        \item \textbf{Name: Snigdha Mondal  } 
        \item \textbf{Roll No: 30059223007 } 
        \item \textbf{Department: Forensic Science and Technology (BSc Forensic Science) } 
        \item \textbf{GitHub Link:} \url{https://github.com/SNIGDHA632}
    \end{itemize}

    \item \textbf{Member 3:} 
    \begin{itemize}[leftmargin=1.5cm]
        \item \textbf{Name: Brithika Das } 
        \item \textbf{Roll No: 30001223062  } 
        \item \textbf{Department: BCA(Bachelor of Computer Application) } 
        \item \textbf{GitHub Link:} \url{https://github.com/brithika-das-04}
    \end{itemize}

    \item \textbf{Member 4:} 
    \begin{itemize}[leftmargin=1.5cm]
        \item \textbf{Name: Subrata Mondal } 
        \item \textbf{Roll No: 30054623018 } 
        \item \textbf{Department: BSc in IT (Artificial Intelligence) } 
        \item \textbf{GitHub Link:} \url{https://github.com/Subrata-crypto}
    \end{itemize}

    \item \textbf{Member 5:} 
    \begin{itemize}[leftmargin=1.5cm]
        \item \textbf{Name: Soumili Das } 
        \item \textbf{Roll No: 30001223017 } 
        \item \textbf{Department: BCA(Bachelor of Computer Application) } 
        \item \textbf{GitHub Link:} \url{https://github.com/Soumilidas1234}
\end{itemize}
\newpage

\section*{Acknowledgement}

We would like to express our deepest gratitude to our course instructor, Sir Ayan Ghosh , for their continuous support, guidance, and invaluable insights throughout this Skill Enhancement Course (SEC). Their expert advice and teaching methods have enhanced our understanding of Git and collaborative tools significantly, enabling us to efficiently complete this group project.

We would also like to thank our group members for their dedication, effort, and teamwork in ensuring the success of this assignment. Each member contributed to the project by sharing ideas, engaging in productive discussions, and completing the required tasks efficiently.

A special mention goes to the Group Leader, Srinjay Nag , for their exemplary leadership in organizing the group, setting up the Git repository, and coordinating the contributions of all members.

Finally, we would like to acknowledge the resources and tools provided by GitHub, which allowed us to seamlessly collaborate and manage the project's workflow in a professional manner. 

This project has not only helped us understand the importance of version control and LaTeX documentation but has also provided us with practical experience that will be valuable in our future academic and professional endeavors.

Thank you.\\ \\ \\ \\
Signature
\newpage 

\vspace{1cm}

\section*{Table of Contents}
\tableofcontents

\vspace{1cm}

\newpage

% Section for each lab entry
\section*{Git Assignment}
\section{Lab 1: Calculator Program using C - (Brithika Das)}

\subsection{Objective}
The objective of this lab is to develop a basic calculator program using the C programming language. The calculator will perform simple arithmetic operations like addition, subtraction, multiplication, and division based on user input.

\subsection{Program Overview}
The calculator program is designed to:
\begin{itemize}
    \item Accept two numbers from the user.
    \item Prompt the user to select an arithmetic operation (Addition, Subtraction, Multiplication, Division).
    \item Perform the selected operation.
    \item Display the result of the operation to the user.
\end{itemize}

The program includes error handling to manage division by zero and other invalid inputs.

\subsection{Code Implementation}
The following is the C code for the calculator program:

\begin{verbatim}
#include <stdio.h>

int main() {
    char operator;
    double num1, num2, result;

    printf("Enter an operator (+, -, *, /): ");
    scanf("%c", &operator);

    printf("Enter two operands: ");
    scanf("%lf %lf", &num1, &num2);

    switch(operator) {
        case '+':
            result = num1 + num2;
            break;
        case '-':
            result = num1 - num2;
            break;
        case '*':
            result = num1 * num2;
            break;
        case '/':
            if (num2 != 0)
                result = num1 / num2;
            else {
                printf("Error! Division by zero.\n");
                return -1;
            }
            break;
        default:
            printf("Error! Operator is not correct\n");
            return -1;
    }

    printf("Result: %.2lf\n", result);
    return 0;

    \subsection{Compiling and Running the Program}
To compile and run the calculator program:
\begin{enumerate}
    \item Open a terminal or command prompt.
    \item Navigate to the directory where the C file is located.
    \item Compile the program using a C compiler (e.g., GCC):
    \begin{verbatim}
    gcc calculator.c -o calculator
    \end{verbatim}
    \item Run the compiled program:
    \begin{verbatim}
    ./calculator
    \end{verbatim}
\end{enumerate}

\subsection{Adding the Calculator Program to GitHub Repository}
To add this calculator program to a GitHub repository, follow these steps:

\subsubsection{Step 1: Initialize a Local Git Repository}
\begin{enumerate}
    \item Open the terminal and navigate to the directory where your \texttt{calculator.c} file is located.
    \item If you haven't already, initialize a Git repository in that directory:
    \begin{verbatim}
    git init
    \end{verbatim}
    This command creates a new Git repository in the current directory.
\end{enumerate}

\subsubsection{Step 2: Add the File to the Repository}
\begin{enumerate}
    \item Add the \texttt{calculator.c} file to the staging area:
    \begin{verbatim}
    git add calculator.c
    \end{verbatim}
    This command stages the file, indicating that you want to include it in the next commit.
\end{enumerate}

\subsubsection{Step 3: Commit the Changes}
\begin{enumerate}
    \item Commit the file to the repository with a meaningful message:
    \begin{verbatim}
    git commit -m "Add calculator program in C"
    \end{verbatim}
\end{enumerate}

\subsubsection{Step 4: Push the Changes to GitHub}
\begin{enumerate}
    \item Link your local repository to a remote GitHub repository:
    \begin{verbatim}
    git remote add origin https://github.com/yourusername/your-repo-name.git
    \end{verbatim}
    \item Push the changes to the GitHub repository:
    \begin{verbatim}
    git push -u origin master
    \end{verbatim}
\end{enumerate}

\subsubsection{Step 5: Verify the Upload}
\begin{enumerate}
    \item Go to your GitHub repository URL in a web browser.
    \item Verify that the \texttt{calculator.c} file is listed and accessible in the repository.
\end{enumerate}

}
\end{verbatim}
\newpage 

\section{Symbol Mind Reading Java Application}

\subsection{Description}
This Java AWT application is a simple graphical program that simulates a mind-reading trick. The user is prompted to think of any two-digit number, reverse the digits, and find the difference between the original and reversed numbers. The user then finds the resulting number in a grid of symbols, each labeled with a number from 0 to 98.

The twist of the program is that all numbers divisible by 9 share the same symbol, which is randomly generated each time the program runs. This symbol is eventually revealed as the "mind-read" symbol when the user clicks the \texttt{Submit} button.

\subsection{Features}
\begin{itemize}
    \item \textbf{Grid of Symbols:} The main window displays a grid of 99 symbols, each paired with a number from 0 to 98.
    \item \textbf{Random Special Symbol:} A random symbol is assigned to all positions in the grid that are divisible by 9.
    \item \textbf{Instructional Message:} The application provides a brief message at the top of the window that guides the user through the mental trick.
    \item \textbf{Submit Button:} Once the user is ready, they click the \texttt{Submit} button to reveal the special symbol in a refreshed window.
\end{itemize}

\subsection{How It Works}
\begin{enumerate}
    \item The user is instructed to think of a two-digit number, reverse its digits, and subtract the smaller number from the larger number.
    \item The user then finds the result in the grid of symbols and memorizes the corresponding symbol.
    \item When the user clicks the \texttt{Submit} button, the application clears the grid and displays the special symbol associated with all multiples of 9, "reading the user's mind."
\end{enumerate}

\subsection{How to Run}
To run the program:
\begin{enumerate}
    \item Compile the Java file using \texttt{javac SymbolApp.java}.
    \item Run the compiled class using \texttt{java SymbolApp}.
    \item The application window will appear, and the user can follow the on-screen instructions.
\end{enumerate}

\subsection{Customization}
\begin{itemize}
    \item The special symbol is generated randomly at the start of the application. You can modify the range of ASCII characters used for generating the symbol in the code if desired.
    \item The grid layout and other UI elements are customizable through the \texttt{GridLayout} and other layout managers used in the AWT framework.
\end{itemize}

\subsubsection{Button Customization}
The button has been customized as follows:

\begin{verbatim}
// Original Button Setup
submitButton = new Button("Chin Tapak Dum Dum");
submitButton.setPreferredSize(new Dimension(250, 60)); // Make the button larger
submitButton.setFont(new Font("Serif", Font.BOLD | Font.ITALIC, 20)); // Change font style
submitButton.setBackground(Color.RED); // Set background color
submitButton.setForeground(Color.WHITE); // Set text color
submitButton.setCursor(new Cursor(Cursor.HAND_CURSOR)); // Change cursor when hovering
\end{verbatim}

These modifications include changing the button's label to \texttt{"Chin Tapak Dum Dum"}, resizing the button, adjusting the font style, and altering the button's color scheme to enhance its appearance and usability.

 \end{itemize}
 \newpage

 \section{ Git Branching, Merging, and Conflict Resolution - (Snigdha Mondal)}

\textbf{Task:} Demonstrate proficiency in Git branching, merging, and conflict resolution in a step-by-step process.

\subsection*{Procedure}
\begin{enumerate}[label=\arabic*.]
    \item \textbf{Create a GitHub Repository:}
    \begin{itemize}
        \item Create a new repository called \texttt{git-advanced} on GitHub.
    \end{itemize}
    
    \item \textbf{Clone the Repository:}
    \begin{itemize}
        \item Clone the repository to your local machine using the command: \texttt{git clone <repository-url>}
    \end{itemize}
    
    \item \textbf{Create and Switch to a New Branch (feature-1):}
    \begin{itemize}
        \item Use the command \texttt{git checkout -b feature-1} to create and switch to a new branch named \texttt{feature-1}.
    \end{itemize}
    
    \item \textbf{Add and Commit Changes on feature-1:}
    \begin{itemize}
        \item Create a file \texttt{shared.txt} and add the content:
        \begin{verbatim}
This is a shared file.
Line 1: Original text.
Line 2: Original text.
\end{verbatim}
        \item Stage and commit the changes: \texttt{git commit -m "Add shared.txt with original text"}
    \end{itemize}
    
    \item \textbf{Push the Branch to GitHub:} 
    \begin{itemize}
        \item Push the \texttt{feature-1} branch to GitHub: \texttt{git push origin feature-1}
    \end{itemize}

    \item \textbf{Create Another Branch (feature-2):}
    \begin{itemize}
        \item Switch to \texttt{feature-2} branch using the command: \texttt{git checkout -b feature-2}
    \end{itemize}

    \item \textbf{Modify the Shared File on feature-2:}
    \begin{itemize}
        \item Modify the second line of \texttt{shared.txt}:
        \begin{verbatim}
Line 2: Modified text in feature-2.
        \end{verbatim}
        \item Stage and commit the changes: \texttt{git add shared.txt}
        \item Commit with a descriptive message: \texttt{git commit -m "Modify Line 2 in feature-2"}
        \item Push the changes: \texttt{git push origin feature-2}
    \end{itemize}

    \item \textbf{Switch Back to feature-1 and Modify:}
    \begin{itemize}
        \item Switch back to \texttt{feature-1} using: \texttt{git checkout feature-1}
        \item Modify the second line on \texttt{shared.txt}:
        \begin{verbatim}
Line 2: Modified text in feature-1.
        \end{verbatim}
\item Stage and commit the changes: \texttt{git add shared.txt}
        \item Commit with a descriptive message: \texttt{git commit -m "Modify Line 2 in feature-1"}
        \item Push the changes: \texttt{git push origin feature-1}
    \end{itemize}

    \item \textbf{Merge feature-1 into main:}
    \begin{itemize}
        \item Switch to the \texttt{main} branch: \texttt{git checkout main}
        \item Merge the \texttt{feature-1} branch into the \texttt{main} branch: \texttt{git merge feature-1}
        \item Push the merged changes: \texttt{git push origin main}
    \end{itemize}

    \item \textbf{Merge feature-2 and Handle Conflict:}
    \begin{itemize}
        \item Merge \texttt{feature-2} into \texttt{main}: \texttt{git merge feature-2}
        \item Git will notify you of a merge conflict in \texttt{shared.txt}.
        \item Open \texttt{shared.txt} in your text editor and resolve the conflict by choosing the appropriate changes or combining them.
        \item After resolving, stage the resolved file: \texttt{git add shared.txt}
        \item Commit the merge: \texttt{git commit -m "Merge feature-2 into main and resolve conflicts"}
        \item Push the resolved main branch to GitHub: \texttt{git push origin main}
    \end{itemize}

    \item \textbf{Clean Up Branches:}
    \begin{itemize}
        \item Delete both \texttt{feature-1} and \texttt{feature-2} branches locally:
        \begin{verbatim}
git branch -d feature-1
git branch -d feature-2
        \end{verbatim}
        \item Delete the branches on GitHub:
        \begin{verbatim}
git push origin --delete feature-1
git push origin --delete feature-2
        \end{verbatim}
    \end{itemize}
\end{enumerate}
\newpage

\section*{\LaTeX Assignment}
 \section{ Introduction to \LaTeX -(Soumili Das)}

\subsection{Abstract}
This document serves as a brief introduction to \LaTeX, a widely-used typesetting system. 
    It covers the basics of document structure, formatting, mathematical typesetting, and features 
    like cross-referencing, tables, and figure handling.

\subsection{What is \LaTeX ?}    
\LaTeX\ is a high-quality typesetting system that is widely used for producing scientific and technical
documents. It is not a word processor; instead, \LaTeX\ allows authors to focus on content rather
than design, with an emphasis on logical structure. Developed in the 1980s by Leslie Lamport,
\LaTeX\ builds on Donald Knuth’s \TeX, adding macros to simplify document production.\\


A few key features of \LaTeX\ include:
\begin{itemize}
    \item Automatic generation of bibliographies, indexes, and tables of contents.
    \item Handling complex mathematical notations.
    \item Consistent and professional typography.
    \item Easy cross-referencing of sections, equations, and figures.
\end{itemize}

\subsection{Document Structure}
Every \LaTeX\ document begins with a document class declaration, which determines the overall
layout. Some common classes are \texttt{article}, \texttt{report}, and \texttt{book}. The document itself is enclosed
between \verb|\begin{document}| and \verb|\end{document}|.\\


Here’s an example of a very simple \LaTeX\ document:
\begin{verbatim}
\documentclass{article}
\begin{document}
    \title{Sample Document}
    \author{Author Name}
    \date{\today}
    \maketitle
    \section{Introduction}
    This is the introduction to our document.
\end{document}
\end{verbatim}

You can add sections, subsections, and paragraphs to structure your document logically. \LaTeX\
automatically numbers sections and subsections for you.

\subsubsection{The Preamble}
The preamble is the part of the document before \verb|\begin{document}|. Here, you can load packages
to extend \LaTeX's functionality, set global formatting options, and define custom commands.\\


Example of including a package:
\begin{verbatim}
\usepackage{amsmath} % For advanced math typesetting
\end{verbatim}

\subsection{Mathematical Typesetting}
\LaTeX\ excels at typesetting mathematical equations, both inline and as standalone expressions. Inline math is written within \verb|$| symbols, like this: $a^2 + b^2 = c^2$. For larger expressions, the \texttt{equation} environment is used.\\


For example, here’s the quadratic formula:

\[
x = \frac{-b \pm \sqrt{b^2 - 4ac}}{2a} 
\]

\subsubsection{Dotted Expressions}
Dots are often used to indicate that a pattern continues. You can use different types of dots for
various purposes in mathematical typesetting:
\[
1 + 2 + 3 + \cdots + n = \frac{n(n+1)}{2}
\]
\[
A = \begin{pmatrix}
a_{11} & a_{12} & \cdots & a_{1n} \\
a_{21} & a_{22} & \cdots & a_{2n} \\ 
\vdots & \vdots & \ddots & \vdots \\
a_{m1} & a_{m2} & \cdots & a_{mn} \\
\end{pmatrix}
\]
In the matrix example, \vdots represents\ vertical\ dots,\ddots represent\ diagonal\ dots, \cdots represents\ horizontal\ dots. 

\subsection{Inserting Figures and Tables}
Including images and tables is simple in \LaTeX. For figures, you use the \texttt{figure} environment and
the \verb|\includegraphics| command.

Here is an example of how to insert a figure:
\begin{verbatim}
\begin{figure}
    \centering
    \includegraphics[width=0.5\linewidth]{image1.png}
    \caption{Enter Caption}
    \label{fig:enter-label}
\end{figure}
\end{verbatim}
        

Tables are created using the \texttt{table} and \texttt{tabular} environments. For example:
\begin{verbatim}
    \begin{table}[h]
    \centering
    \begin{tabular}{|c|c|c|}
        \hline
        Column 1 & Column 2 & Column 3 \\
        \hline
        1 & 2 & 3 \\
        4 & 5 & 6 \\
        \hline
    \end{tabular}
    \caption{A simple table.}
    \label{tab:example}
\end{table}
\end{verbatim}

\begin{table}[h]
    \centering
    \begin{tabular}{|c|c|c|}
        \hline
        Column 1 & Column 2 & Column 3 \\
        \hline
        1 & 2 & 3 \\
        4 & 5 & 6 \\
        \hline
    \end{tabular}
    \caption{A simple table.}
    \label{tab:example}
\end{table}

\subsection{Cross-Referencing and Citations}
\LaTeX\ provides robust tools for cross-referencing sections, equations, and figures. For example, to reference Figure 1, use the \verb|\ref| command. You can also automatically generate a bibliography by
using \verb|BibTeX| or \verb|BibLaTeX|.

\subsection{Compiling the Document}
To see the formatted output of your \LaTeX\ document, you need to compile it. The most common
compiler is \texttt{pdflatex}, which generates a PDF from your \LaTeX\ source code. Online editors like \textbf{ Overleaf} provide a user-friendly interface for compiling and editing \LaTeX\ documents.

\subsection{Conclusion}
This brief introduction to \LaTeX\ covers its fundamental concepts and features. As you become more comfortable, you can explore advanced topics such as custom commands, style files, and the extensive array of packages available for \LaTeX. It is a powerful tool that ensures professional, consistent document formatting, making it a favorite for researchers, scientists, and academic writers.
\newpage
\section{Create a LaTeX Document -(Soumili Das)}

\textbf{Task:} Your task is to create a LaTeX document. The document should be formatted to look exactly like the provided attachment.

\subsubsection*{Procedure}
\begin{enumerate}[label=\arabic*.]
    \item \textbf{Prepare the Document:}
    \begin{itemize}
        \item Open your LaTeX editor (Overleaf, TeXShop, or similar).
        \item Create a new document and ensure it matches the formatting of the given attachment.
    \end{itemize}
    
    \item \textbf{Document Naming Convention:}
    \begin{itemize}
        \item When your LaTeX document is complete, name it according to the following rule: \texttt{Rollno\_DeptName\_Firstname.tex}.
        \item This file should contain your LaTeX code.
    \end{itemize}
    
    \item \textbf{Create Output and Zip Files:}
    \begin{itemize}
        \item Compile your LaTeX document to generate the output file in PDF format.
        \item Include an image file (either \texttt{.png} or \texttt{.jpg}) that should be inserted into your LaTeX document.
        \item Create a zip file containing the following three files:
        \begin{itemize}
            \item Your source code: \texttt{Rollno\_DeptName\_Firstname.tex}
            \item Your compiled output: \texttt{Rollno\_DeptName\_Firstname.pdf}
            \item Your image file: \texttt{.png} or \texttt{.jpg}
        \end{itemize}
        \item Name the zip file as \texttt{Rollno\_DeptName\_Firstname.zip}.
    \end{itemize}
\end{enumerate}

\subsubsection*{Deliverables}
You need to upload the zip file containing:
\begin{itemize}
    \item Your LaTeX source code (\texttt{.tex})
    \item The compiled PDF output
    \item An image file (\texttt{.png} or \texttt{.jpg})
\end{itemize}
\newpage

\section{Create a CV Using LaTeX -(Srinjay Nag)}

\textbf{Task:} Create a CV using a LaTeX document.

\subsubsection*{Procedure}
\begin{enumerate}[label=\arabic*.]
    \item \textbf{Outline Your CV Content:}
    \begin{itemize}
        \item Include your name, contact details, and a professional summary.
        \item List academic qualifications, work experience, skills, projects, and certifications.
    \end{itemize}

    \item \textbf{Decide on the Structure and Layout:}
    \begin{itemize}
        \item Organize the CV into sections such as Personal Information, Experience, Education, etc.
    \end{itemize}

    \item \textbf{Choose a LaTeX Template:}
    \begin{itemize}
        \item Select a template that suits your style from Overleaf or a LaTeX library.
    \end{itemize}

    \item \textbf{Customize the Template:}
    \begin{itemize}
        \item Edit the template with your personal content (experience, qualifications, etc.).
    \end{itemize}

    \item \textbf{Adjust Formatting:}
    \begin{itemize}
        \item Ensure consistency in fonts and section headings.
    \end{itemize}

    \item \textbf{Proofread and Finalize:}
\begin{itemize}
        \item Review for any errors or formatting issues.
        \item Ensure alignment and organization of sections.
    \end{itemize}

    \item \textbf{Compile and Export:}
    \begin{itemize}
        \item Compile the LaTeX document and export it as a PDF for sharing.
    \end{itemize}
\end{enumerate}

\subsubsection*{Deliverables}
You need to upload the final CV in PDF format, created using LaTeX.

\end{document}   